\chapter{Konstruktion}
Für die Konstruktion des \textit{EarthROVERs} wurden drei \textit{Mindstorms education} Kästen und ein \textit{Zusatzkasten} verwendet. Des weiteren wurden einige HiTechnic Sensoren verwendet.

\section{Basis}
Die Basis trägt alle Teile des \textit{EarthROVERs}, daher muss sie relativ groß sein und möglichst stabil. Bei der Basis haben wir uns für eine flache Konstruktion entschieden.

\begin{capfigure}[Basis]
	\includegraphics[width=\textwidth]{images/construction/basis/basis}
\end{capfigure}

Auf der Abbildung ist die Basis mit einem der \textit{NXT-Bricks}, dem \nameref{cha:fahrwerk} und den \nameref{cha:sensoren} 

\subsection{Sensorturm}
Für den \textit{HiTechnic Compass Sensor} und den \textit{HiTechnic Gyro Sensor} wurde ein kleiner Turm gebaut. An der Spitze des Turmes sind die beiden Sensoren befestigt.

\begin{capfigure}[Sensorturm]
	\includegraphics[width=5cm]{images/construction/basis/turm}
\end{capfigure}

Der Turm wurde möglichst nah am Drehpunkt des Roboters befestigt. 

\subsection{Sensoren}
\label{cha:sensoren}
Die Sensoren wurden an der Vorderseite der Basis angebracht. 

\begin{capfigure}[Sensoren]
	\includegraphics[width=\textwidth]{images/construction/basis/sensoren}
\end{capfigure}

Der \textit{HiTechnic Infrared Seeker} wurde in der Mitte angebracht. Der \textit{Ultraschallsensor} wurde an der linken Seite des \textit{EarthROVERs} angebracht.

\section{Fahrwerk}
\label{cha:fahrwerk}
Das Fahrwerk besteht aus vier Ketten. Auf jeder Seite des \textit{EarthROVERs} befinden sich je zwei Ketten.

\begin{capfigure}[Fahrwerk]
	\includegraphics[width=\textwidth]{images/construction/basis/kette}
\end{capfigure}

\section{Greifarm}
\TODO{Beschreibung + Bild}