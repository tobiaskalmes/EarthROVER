\chapter{Nicht Umgesetzte Funktionen}
\section{Pi-zu-Brick-Kommunikation}
\label{lbl:ptbcomm}
Da wir nicht wie anfangs geplant die \textit{NXT-Bricks} via USB mit dem RaspberryPi verbinden konnten blieb uns an dieser Stelle nur die Kommunikation via Bluetooth. Hierfür fehlt uns allerdings die Hardware(ein Bluetooth-Dongle). Daher mussten wir diesen Teil streichen. Ohne diesen Teil funktionieren auch andere Teile der Vision nicht.

\section{Webservices}
Es war geplant die Kommunikation zwischen dem RaspberryPi und dem Web-Interface über RESTful Webservices zu regeln. Dafür hätten wir auf dem RaspberryPi einen kleinen Webserver laufen lassen(einen Eigenen) und würden dort zum einen die Sensordaten anbieten als auch Befehle für den \textit{EarthROVER} annehmen. Die Webservices selbst waren mit Jetty geplant. Die grundsätzliche Umsetzung der Webservices inklusive Tests wurde durchgeführt. Hier gab es keine Probleme. Da wie in \nameref{lbl:ptbcomm} beschrieben keine Kommunikation zwischen RaspberryPi und dem \textit{EarthROVER} möglich ist können die Webservices auch keine Daten liefern bzw. keine Befehle weitergeben.

\section{Webseite}
Die Webseite sollte, die aktuellen Sensordaten unterhalb des aktuellen Lifestreams anzeigen. Als Sprache wäre für die Darstellung der Daten PHP genutzt worden und die akutellen Werte, der Sensoren, pro Sekunde ausgelesen worden. Der Webserver war bereits schon lauffähig und das Livebild ist über den Browser sichtbar. Da jedoch keine Kommunktion zustande kam, wurden nur Dummy-Werte genutzt und diese später auf der Webseite entfernt, außer dem aktuellen Livebild.

\section{Fernsteuerung über Webinterface}
Wäre die Kommunikation geglückt, wären auf der Webseite, unterhalb der Sensordaten, Buttons mit Funktionen zur Fernsteuerung realisiert worden. Diese hätten über php und den Webservice, die direkte Steuerung des Rover ermöglicht. Gedacht waren dabei die Buttons Vorwärts, Rückwärts, Links, Rechts, Heim fahren und Ball suche fortsetzen. Beim drücken der Tasten Vorwärts, Rückwärts, Links und Rechts wäre das autnome fahren direkt beendet worden.

\section{Ball finden}
Da uns in der Endphase der Infrarotball gefehlt hat war es uns nicht mehr möglich die Suche einzubinden. Die Überlegung wie die Suche umgesetzt werden kann wurden gemacht. Da wir aber keine Möglichkeit hatten die Umsetzung zu testen haben wir es dabei belassen die Umsetzung zu beschreiben.

\section{Brick-zu-Brick-Kommunikation}
Bei unserer ursprünglichen Idee hätten wir keine Brick-zu-Brick-Kommunikation benötigt. Da es angedacht war, dass der RaspberryPi die gesamte Logik ausführt. Hierfür mangelte es und am Ende dann an der Zeit. Darum wollten wir ein NXT rein zur Remote nutzen. Dieser sollte über Bluetooth befehle an einen anderen senden, zum Beispiel das heben und senken des Armes. Das Programm wurde auch in den Grundzügen implementiert aber leider konnte sich zwischen den Geräten keine Verbindung aufbauen. Aus Zeitgründen wurde dies nicht weiter verfolgt.

