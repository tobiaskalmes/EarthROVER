\chapter{Umsetzung}
\section{Fahren}
Da der \textit{EarthROVER} mit vier Ketten fahren soll bedarf es hier einer besonderen Lösung. Um vier Motoren an einem \textit{NXT-Brick} zu verwenden zu können werden zwei Motoren(auf einer Seite) mit Hilfe von RCX-Motorkabeln verbunden. Hierzu werden drei RCX-Kabel verwendet. Das erste Kabel wird in den NXT gesteckt. An das andere Ende wird ein zweites RCX-Kabel aufgesteckt. Ein drittes Kabel wird verkehrt herum aufgesteckt. Hierdurch werden beide Motoren entgegengesetzt drehen. Da bei unserem Aufbau auf der Basis die Motoren auf einer Seite entgegensetzt sind passt das so genau. 

\TODO{Bild der RCX-Kabel einfügen}

Zu Beachten ist hierbei, dass durch die Benutzung der RXC-Kabel die Motoren keine Tachowerte mehr liefern. Man kann die Motoren als nicht reguliert verwenden und auch keine Tachowerte auslesen.

Dies stellt in unserem Szenario allerdings kein Problem dar, da es für das Fahren bei unserer Konstruktion ausreicht bei einem der vier Motoren in der Lage zu sein den Motor reguliert zu betreiben.

\subsection{Orientierung}
Zur Orientierung verwendet der \textit{EarthROVER} Kompass-werte. Um Fehlmessungen des Kompass abzuschwächen wurde ein Kalmanfilter verwendet. Für den Kalmanfilter wird neben dem \textit{HiTechnic Compass Sensor} noch ein \textit{HiTechnic Gyro Sensor} verwendet.

Neben dem eigentlichen Kalmanfilter werden die gelesenen Werte der beiden Sensoren vor dem Filtern über einen kurzen Zeitraum gemittelt um Ausreißer auszugleichen.

\subsection{Weg merken}
Damit der \textit{EarthROVER} sich den Weg möglichst präzise Merken kann haben wir uns dazu entschieden, dass der \textit{EarthROVER} sich entweder geradlinig bewegt oder auf der Stelle dreht. Er kann also keine Kurven fahren. Dies hat den Vorteil, dass wir präzise Werte haben um den Weg aufzuzeichnen.

Der \textit{EarthROVER} merkt sich zwei Arten von Bewegungen:
\begin{capitemize}[Bewegungsarten]
	\item Geradlinig Fahren: Umdrehungen und Fahrtrichtung wird aufgezeichnet
	\item Auf der Stelle drehen: Mit Hilfe des gefilterten Kompasses wird die Differenz zwischen Startwinkel und Endwinkel und die Drehrichtung aufgezeichnet.
\end{capitemize}

\subsection{Zurück fahren}
Um den \textit{EarthROVER} wieder zu seinem Startpunkt zu bringen haben wir zwei Optionen vorgesehen. Umsetzen konnten wir nur die \nameref{lbl:simplemethod}.

\subsubsection{Simple Methode}
\label{lbl:simplemethod}
Bei der simplen Methode fährt der \textit{EarthROVER} den Weg genau so wieder zurück wie er ihn gefahren ist. Dazu geht er einfach die aufgezeichneten Schritte in umgekehrter Reihenfolge durch und führt die umgekehrte Aktion durch(Anstelle von Vorwärts Rückwärts, etc.).

\subsubsection{Direkte Methode}
Bei der direkten Methode sollte der \textit{EarthROVER} den direkten Weg zum Ausgangspunkt fahren. Um dies zu erreichen wird nach jeder zurückgelegten Fahrtbewegung(Vorwärts bzw. Rückwärts) der direkte Weg berechnet. Hierzu wird sich der allgemeine Kosinussatz zur Hilfe genommen.

\begin{capdefinition}[Allgemeiner Kosinussatz]	
	$c^2 = a^2 + b^2 - 2ab*cos(\gamma)$
	$b^2 = a^2 + c^2 - 2ac*cos(\beta)$
\end{capdefinition}

\begin{capfigure}[Kosinussatz]
	\includegraphics[width=5cm]{images/implementation/cosinus}
\end{capfigure}

Die Berechnung beginnt ab der zweiten Fahrbewegung. $A$ stellt den Startpunkt dar. Die erste Fahrbewegung ist $b$, die erste Drehbewegung ist $\gamma$ und die zweite Fahrbewegung ist $a$. Um aus dieser Position zurück zu gelangen muss zum einen $c$ und zum Anderen $\beta$ bekannt sein.

Um $c$ zu berechnen müssen wir lediglich die obere Formel wie folgt umstellen:

$c = \sqrt[2]{a^2 + b^2 - 2ab*cos(\gamma)}$

Damit haben wir unser $c$, jetzt fehlt noch unser $\beta$. Hierfür wird die zweite Formel verwendet. Wir stellen diese wie folgt um:

$\beta = arcos((a^2 + c^2 - b^2)2ac)$

Nun können wir auch unser $\beta$ berechnen. Damit haben wir alles um zum Ausgangspunkt zurück zu fahren.

Für den nächsten Schritt werden dann wie folgt die Werte zugewiesen:\\
\begin{tabular}[htbp]{|l|l|}
	Neue Werte & alte Werte \\
	\hline
	$b$ & $c$ \\
	\hline
	$a$ & Fahrstrecke des nächsten Schritts \\
	\hline
	$\gamma$ & \\
\end{tabular}

\section{Ball finden}

\section{Greifarm}

\section{Pi}