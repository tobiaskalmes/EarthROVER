%Needed commands
\newcommand{\createifnotexists}[1]
{
	\newread\check 
	\immediate\openin\check #1 
	\ifeof\check
		\newwrite\results 
		\immediate\openout\results=#1
		\immediate\closeout\results
	\else
		\closein\check
	\fi
}

\newcommand{\TODO}[1]{\textcolor{red}{TODO: #1}}

%new defines
\newrefformat{cha}{Kapitel~\ref{#1}}
\newrefformat{sec}{Abschnitt~\ref{#1}}
\newrefformat{fig}{Abbildung~\ref{#1}}
\newrefformat{tab}{Tabelle~\ref{#1}}
\newrefformat{list}{Liste~\ref{#1}}
\newrefformat{exmp}{Beispiel~\ref{#1}}
\newrefformat{def}{Definition~\ref{#1}}
\newrefformat{lst}{Listing~\ref{#1}}

\newtheorem{example}{Beispiel}[chapter]
\newtheorem{definition}{Definition}[chapter]
\DeclareCaptionType{items}[Liste] 
\renewcommand{\lstlistingname}{Listing}

%Current caption
\newcommand*{\currentcaption}{}

%Captionized environments
\newenvironment{capenumerate}[1][\currentcaption]
{%BEGIN%	
	\renewcommand*{\currentcaption}{#1}
	\begin{center}
		\captionsetup{type=items}
		\begin{enumerate}
}{%END%
		\end{enumerate}
		\caption{\currentcaption}	
		\label{list:\currentcaption}
		\addcontentsline{loi}{items}{\numberline{\theitems}\currentcaption}	
	\end{center}	
}

\newenvironment{capitemize}[1][\currentcaption]
{%BEGIN%	
	\renewcommand*{\currentcaption}{#1}
	\begin{center}
		\captionsetup{type=items}
		\begin{itemize}
}{%END%
		\end{itemize}
		\caption{\currentcaption}	
		\label{list:\currentcaption}
		\addcontentsline{loi}{items}{\numberline{\theitems}\currentcaption}	
	\end{center}	
}

\newenvironment{captabular}[2][\currentcaption]
{%BEGIN%
	\renewcommand*{\currentcaption}{#1}
	\begin{table}[htbp]
		\captionsetup{type=table}
		\begin{center}
		\begin{tabular}{#2}
}{%END%
		\end{tabular}
		\end{center}
		\caption{\currentcaption}
		\label{tab:\currentcaption}
	\end{table}
}

\newenvironment{captabular*}[3][\currentcaption]
{%BEGIN%
	\renewcommand*{\currentcaption}{#1}
	\begin{table}[htbp]
		\captionsetup{type=table}
		\begin{center}
		\begin{tabular*}{#2}{#3}
}{%END%
		\end{tabular*}	
		\end{center}
		\caption{\currentcaption}
		\label{tab:\currentcaption}
	\end{table}
}

%Doesn't work yet
\newenvironment{captabularx}[3][\currentcaption]
{%BEGIN%
	\renewcommand*{\currentcaption}{#1}
	\begin{table}[htbp]
		\captionsetup{type=table}
		\tabularx{#2}{#3}
}{%END%
		\endtabularx
		\caption{\currentcaption}
		\label{tab:\currentcaption}
	\end{table}
}

\newenvironment{capfigure}[1][\currentcaption]
{%BEGIN%
	\renewcommand*{\currentcaption}{#1}	
		\begin{figure}[hbtp]
			\captionsetup{type=figure}
				\begin{center}
}{%END%
				\end{center}
			\caption{\currentcaption}
			\label{fig:\currentcaption}		
		\end{figure}
}

\newenvironment{capexample}[1]
{%BEGIN%
	\begin{example}[#1]
		\label{exmp:#1}
}{%END%
	\end{example}
}

\newenvironment{capdefinition}[1]
{%BEGIN%
	\begin{definition}[#1]
		\label{def:#1}
}{%END%
	\end{definition}
}

\lstnewenvironment{caplisting}[2]  
{%BEGIN%
	\singlespacing  
	\lstset{label=lst:#1, caption=#1} %Caption and Label
	%\lstset{basicstyle=\small\ttfamily, tabsize=4}
	%\lstset{breaklines=true, breakatwhitespace=true, postbreak=\space}
	%\lstset{numbers=left, numberstyle=\tiny, numbersep=5pt}
	%\lstset{keywordstyle=\color{blue}}
	%\lstset{stringstyle=\itshape\color{green}, showstringspaces=false}
	%\lstset{commentstyle=\itshape\color{red}}	
	\lstset{#2}
}{} 

%References to captionized environments
\newcommand{\caplistref}[1]{\prettyref{list:#1}}
\newcommand{\captabularref}[1]{\prettyref{tab:#1}}
\newcommand{\capfigureref}[1]{\prettyref{fig:#1}}
\newcommand{\capexampleref}[1]{\prettyref{exmp:#1}}
\newcommand{\capdefinitionref}[1]{\prettyref{def:#1}}
\newcommand{\caplistingref}[1]{\prettyref{lst:#1}}


    
%Redefine and create lists  
\renewcommand{\listoffigures}{
	\addcontentsline{toc}{chapter}{\listoffiguresname}
	\listoffigures
	\cleardoublepage
}

\renewcommand{\listoftables}{
	\addcontentsline{toc}{chapter}{\listoftablesname}
	\listoftables
	\cleardoublepage
}

\newcommand{\listoflists}{%
	\chapter{Listenverzeichnis}
	\begingroup
		\makeatletter
		\createifnotexists{\jobname.loi}
		\input{\jobname.loi}
		\expandafter\newwrite\csname tf@loi\endcsname
		\immediate\openout \csname tf@loi\endcsname \jobname.loi\relax
	\endgroup
	\cleardoublepage
}

\newcommand{\listoftheorems}[2][Theorems]{
	%\let\cleardoublepage\clearpage
	%\newpage
	\ifnum\value{curr#1ctr}>0
		\chapter{#2}
		\begin{minipage}[t]{\textwidth}	
			\addtolength{\marginparwidth}{-18pt}
			\addtolength{\leftskip}{18pt}
			\addtolength{\textwidth}{-18pt}
			\begin{minipage}[t]{\textwidth}
				\listtheorems{#1}
			\end{minipage}
		\end{minipage}
		\cleardoublepage	
	\fi
}

\newcommand{\listofexamples}{
	\listoftheorems[example]{Beispielverzeichnis}
	\cleardoublepage
}

\newcommand{\listofdefinitions}{
	\listoftheorems[definition]{Definitionsverzeichnis}
	\cleardoublepage
}

\renewcommand{\lstlistlistingname}{Listingverzeichnis}
\newcommand{\listoflistings}{
	%\ifnum\value{lstlisting}>0
		\lstlistoflistings
		\cleardoublepage
	%\fi
}

\newcommand{\listofabbreviations}{
	\printglossary[
			title=Abk�rzungsverzeichnis,
			toctitle=Abk�rzungsverzeichnis,
			style=altlist,
			type=\acronymtype
	]
	\cleardoublepage
}

\newcommand{\listofglossaryentries}{
	\printglossary[
			title=Glossar,
			toctitle=Glossar,
			style=altlist
	]
	\cleardoublepage
}

\newcommand{\listofbibliographyentries}{
	%\nocite{*}
	\printbibliography[
			title=Literaturverzeichnis,
			heading=bibintoc
	]
	\cleardoublepage
}

